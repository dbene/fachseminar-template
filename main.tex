\documentclass[12pt,twocolumn,twoside]{conference}
\usepackage{german}
\title{Hier kommt ihr viel zu langer Titel hin, darf auch gern über mehrere Zeilen gehen}
\author{Autor Eins, Rotua Zwei}
\begin{document}
\twocolumn[
  \begin{@twocolumnfalse}
  \maketitle\thispagestyle{firststyle}
  
    \begin{abstract}
    \vspace{8pt}
      Wer das Paper auf englisch schreiben möchte kommentiert Zeile 2 mit einem prozentzeichen aus oder löscht sie gleich. Wer das Paper auf englisch schreiben möchte kommentiert Zeile 2 mit einem prozentzeichen aus oder löscht sie gleich. Wer das Paper auf englisch schreiben möchte kommentiert Zeile 2 mit einem prozentzeichen aus oder löscht sie gleich.
    \end{abstract}
    \vspace{16pt}
  \end{@twocolumnfalse}
]
\section{Textabschnitt}
\subsection{Unterpunkt 1}
Wer das Paper auf englisch schreiben möchte kommentiert Zeile 2 mit einem prozentzeichen aus oder löscht sie gleich.\footnote{Fußnote} Wer das Paper auf englisch schreiben möchte kommentiert Zeile 2 mit einem prozentzeichen aus oder löscht sie gleich.

\begin{table}[!h]
  \centering 
  \label{tab:table1}
  \begin{tabular}{l|c||r}
    1 & 2 & 3\\
    \hline
    a & b & c\\
  \end{tabular}
  \caption{Caption for the table.}
\end{table}

Wer das Paper auf englisch schreiben möchte kommentiert Zeile 2 mit einem prozentzeichen aus oder löscht sie gleich.

\begin{figure}[H]
\centering
\includegraphics[width=5cm]{fhbielefeld_logo.png}
\caption{Write some caption here}\label{visina8}
\vspace{-12pt}
\end{figure}

Wer das Paper auf englisch schreiben möchte kommentiert Zeile 2 mit einem prozentzeichen aus oder löscht sie gleich.
\subsection{Unterpunkt 2}
Wer das Paper auf englisch schreiben möchte kommentiert Zeile 2 mit einem prozentzeichen aus oder löscht sie gleich.
\subsection{Unterpunkt 3}
Wer das Paper auf englisch schreiben möchte kommentiert Zeile 2 mit einem prozentzeichen aus oder löscht sie gleich. Wer das Paper auf englisch schreiben möchte kommentiert Zeile 2 mit einem prozentzeichen aus oder löscht sie gleich.
\subsection{Zitierbeispiel}
'The structure of the Definition Schema is a representation of the data model of SQL.' \cite{ISO9075-1:2011}

\newpage
\begin{thebibliography}{99}
	
	\bibitem{ISO9075-1:2011}
	ISO/IEC 9075-1: Information technology
	Database languages -SQL- ,Part 1: Framework
	(SQL/Framework), 4. Auflage, ISO Copyright Office,
	Genf 2011
	
	
\end{thebibliography}

\end{document}